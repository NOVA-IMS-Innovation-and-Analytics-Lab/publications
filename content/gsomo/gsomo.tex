\documentclass[parskip=full]{scrartcl}

\pdfoutput=1

\title{G-SOMO \\ \LARGE{An Oversampling Approach based on Self-Organized Map Oversampling and Geometric SMOTE}}

\author{
	Georgios Douzas\(^{1}\), Fernando Bacao\(^{1*}\), Rene Rauch\(^{1}\)
	\\
	\small{\(^{1}\)NOVA Information Management School, Universidade Nova de Lisboa}
	\\
	\small{*Corresponding Author: bacao@novaims.unl.pt}
	\\
	\\
	\small{Postal Address: NOVA Information Management School, Campus de Campolide, 1070-312 Lisboa, Portugal}
	\\
	\small{Telephone: +351 21 382 8610}
}

\usepackage{breakcites}
\usepackage{float}
\usepackage{graphicx}
\usepackage{geometry}
\geometry{
	a4paper,
	total={170mm,257mm},
	left=18mm,
	right=18mm,
	top=8mm,
}
\usepackage{amsmath}
\newcommand{\inlineeqnum}{\refstepcounter{equation}~~\mbox{(\theequation)}}
\usepackage{enumitem}
\usepackage[ruled,vlined]{algorithm2e}
\usepackage{booktabs}
\usepackage{pgfplotstable}
\usepackage{longtable}
\usepackage{tabu}
\usepackage{hyperref}
\date{}

\begin{document}

\maketitle

\begin{abstract}
Traditional supervised machine learning classifiers are challenged to learn highly skewed data 
distributions as they are designed to expect classes to equally contribute to the minimization 
of the classifiers cost function. Moreover, the classifiers design expects equal 
misclassification costs, causing a bias for underrepresented classes. Thus, different 
strategies to handle the issue are proposed by researchers. The modification of the data set 
managed to establish since the procedure is generalizable to all classifiers. Various algorithms 
to rebalance the data distribution through the creation of synthetic instances were proposed in the 
past.  In this paper, we propose a new oversampling algorithm named G-SOMO, a method that is 
inspired by our previous research. The algorithm identifies optimal areas to create artificial 
data instances in an informed manner and utilizes a geometric region during the data generation 
to increase variability and to avoid correlation. Our empirical results on 69 datasets, validated 
with different classifiers and metrics against a benchmark of commonly used oversampling methods 
show that G-SOMO consistently outperforms competing oversampling methods. 
The statistical significance of our results is proven. 
\end{abstract}

\section{Introduction}

Learning from imbalanced datasets remains a challenge for supervised machine learning classifiers 
and therefore needs to be addressed by the research community. Class imbalance refers to 
classification problems where the target class is unequally distributed and therefore algorithms 
trained with imbalanced data tend to be heavily biased towards the majority class \cite{Hoens2013}. 
The paper on hand will present an effective way to deal with imbalanced datasets, the focus hereby is 
set on highly skewed binary datasets. Commonly known domains dealing with these challenges are 
fraud detection, medical diagnosis, risk management, airborne imagery, face recognition and of forecasting 
of ozone levels \cite{Akbani2004}.

Datasets with an unequally distributed target class are identified by their Imbalance Ratio (IR). The class 
represented by most of the instances is named the majority class, while the other class is the minority 
class \cite{Chawla2003}. The imbalance Ratio is defined as the ratio between the majority class and each 
of the minority classes. It is important to notice that the data imbalance is an integral part of the problem 
itself, implying that the cost of a false negative prediction is usually much larger than the cost of a 
false positive. Considering a machine learning based system to make medical decisions, if the model predicts 
a patient to carry a disease, even though the patient is healthy, the wrong decision will be revealed in 
further medical examinations. On the other hand, if the model classifies the patient to be healthy, even 
though the individual carries a disease, tremendous harm is caused \cite{Wan2014LearningTI}. Therefore, 
we can see that the correct classification of the minority class is usually more important than the 
correct prediction of the majority class.

There are two main explanations for the poor ability to learn from imbalanced data sets. The first reason 
to explain this behavior is, that during the training process of the model the instances of the minority 
classes contribute less to the minimization of the algorithms cost function. As a result, the model has 
a higher incentive to classify new instances as majority instances, since there is a higher chance of a 
correct classification. Second, due to the limited number of samples, the model faces difficulties to 
differentiate between outliers and important instances. This implies that the ability to create a decent 
model is not just dependent on the machine learning algorithm and its parameters, but also from the data 
itself and its imbalance ratio.

Unlearned practitioners who trained models on imbalanced data might be lead to false conclusions while 
validating their models based on their accuracy. Accuracy is a measure that is heavily biased towards 
the majority class, the metric might imply a decent score, even though none of the minority classes are 
correctly classified. Assuming a dataset with 99\% majority instances and only 1\% minority instances, the 
model accuracy would still be 99\% without classifying a single minority instance correctly. To address 
this concern, other validation metrics have to be selected that consider the importance of the minority 
class. A more detailed outline of the chosen metrics follows in section 5.1 'Metrics'.

In general, there are three different options to handle an imbalanced data set \cite{Fernandez2013}. 
The first option is the modification of the dataset itself, hereby one can create new synthetic data 
instances to the minority classes by over-sampling or remove existing data instances from the majority 
instances through under-sampling. Both approaches can also be combined to a hybrid approach, the general 
objective of the approaches is to minimize the skewed distribution of the classes within the dataset 
and to shift the classes towards an equal distribution. The second option is to adjust the cost function 
of the supervised learning algorithm. During the training process, the model will be heavily penalized 
for falsely classified minority class instances. The latter implies to modify the training process and 
create explicit algorithms to handle the class imbalance. The third approach focuses on the modification 
of algorithms that reinforce the learning towards the minority class. We decided to tackle the challenge 
of imbalanced datasets by oversampling. The reasons to modify the dataset and not the algorithm is that 
after the dataset modification any supervised machine learning algorithm can be used, without any further 
modification. With our proposed algorithm we focus on oversampling since it provides the advantage that none 
of the majority instances need to be removed and no valuable information is vanished, as it might happen 
through under-sampling. 

The method presented in this paper extends the Self-Organizing Map Oversampling algorithm by making use of the 
Geometric SMOTE algorithm. SOMO is a cluster based algorithm that leverages the Self-Organizing map algorithm 
and managed to outperform related algorithms over various datasets \cite{Douzas2017B}. The algorithm preserves 
the topology while reducing the dimensionality to a two-dimensional representation. The emerging clusters 
are used to create minority instances within, but also between neighboring minority clusters by leveraging 
the SMOTE algorithm. The algorithm presented in this paper utilizes the procedure of the SOMO algorithm and 
replaces the SMOTE algorithm with the Geometric SMOTE algorithm. G-SMOTE generates synthetic samples in a 
geometric region of the input space, around each selected minority instance. G-SMOTE also proved significant 
improvements in the generated data quality \cite{Douzas2017}. Due to the combination of both algorithms, the 
naming choice of the algorithm proposed by us is G-SOMO, Geometric Self-Organizing Map Oversampling.

Following this introduction, we will outline related methods that proved to be efficient in the second chapter. 
Subsequently, we will expound the shortcomings of classifiers and explain our motivation for the oversampling 
method G-SOMO in chapter three. In the fourth chapter, we will discuss the G-SOMO algorithm in detail and 
present each step individually. The following chapter 'Research Methodology‘ covers our experimental pipeline 
to validate the performance of our proposed method. We will present our significant results in chapter six. 
The last chapter will summarize our findings and provides a prospect for further research.

\section{Related work}

The section outlines the current state of the art over-sampling methods. Over-sampling methods generate 
synthetic examples for the minority class and add them to the training set, therefore additional information 
is created. In contrast to over-sampling, under-sampling methods reduce samples of the majority class to 
establish a class balance This implies that information is excluded, which might affect the learning process 
negatively when the data set is small. Both methods have shown to be effective, depending on the problem 
addressed \cite{Chawla2002}. More information on under-sampling methods can be found in \cite{Ganganwar2012} 
and \cite{Yen2006}. Synthetic data instances can be created uninformed, by randomly duplicating minority 
instances or informed, by identifying areas where oversampling is deemed to be most effective \cite{Last2017}. 

The modest form of oversampling is Random Oversampling (ROS). ROS is an uninformed approach, which randomly 
selects minority samples and duplicates them exactly without any selection criteria. The method stands out 
through its simplicity, but proved to increase the risk of overfitting enormously, since the same information 
is used multiple times during the training process, no instances that clarify the decision boundaries are created. 

The most popular approach among practitioners in the domain of oversampling is SMOTE, introduced in 2002. The 
algorithm chooses a random minority instance and identifies its k nearest neighboring minority instances. The 
parameter k is chosen beforehand. A synthetic instance is created on a random point along a line segment joining 
the selected minority instance and one of its neighbors \cite{Chawla2002}.  Depending upon the amount of 
over-sampling required, neighbors from the k nearest neighbors are randomly chosen. Figure \ref{fig:Schubach} 
constitutes the process of SMOTE.

\begin{figure}[H]
	\centering
	\includegraphics[width=16.5cm, height=5cm, keepaspectratio]{./resources/SMOTE_Dummy.png}
	\captionbelow{SMOTE Algorithm, inspired by \cite{Schubach}}
	\label{fig:Schubach}
\end{figure}

Compared to ROS, the synthetic samples are more generalizable and therefore the high risk of overfitting is 
reduced. However, SMOTE has several drawbacks. First, the algorithm randomly selects a minority instance 
for oversampling with uniform probability. Hereby it manages to tackle between-class imbalance, while 
within-class imbalance is ignored. Areas that were highly populated by minority samples will expand, while 
smaller areas of minority samples will remain sparse \cite{Prati2004}. Second, the algorithm promotes the 
generation of noisy samples. When selecting a noisy sample and the linear interpolation to the nearest 
neighbor is created, a new noisy sample might be created, due to the great distance between them. It can 
not distinguish overlapping class regions from safe areas \cite{Bunkhumpornpat2009}. 

To tackle the problems of SMOTE several modifications were created. SMOTE + Edited Nearest Neighbor removes any 
misclassified instances after the creation of synthetic samples, by applying the edited nearest neighbor 
rule \cite{Maria2004}. Safe-Level SMOTE applies a weight degree to differentiate between noisy and safe instances 
\cite{Bunkhumpornpat2009}. Furthermore, G-SMOTE manages to extend the linear interpolation of SMOTE by 
introducing a geometric region where the data generation process occurs. Thereby, the algorithm increases the 
data variability and prevents additional correlation between the created samples \cite{Douzas2017}. Further 
algorithms like Borderline-SMOTE \cite{Han2005}, MWMOTE (Majority Weighted Minority Oversampling Technique 
for Imbalanced Data Set Learning) \cite{Barua2014}, ADASYN and its variation KernelADASYN \cite{Tang2015} 
are trying to avoid noisy samples and focus on hard to learn instances. Hereby, the borderline samples of 
the classes are identified to use the informative minority class instances.

The prior algorithms focus exclusively on between-class imbalance \cite{Nekooeimehr2015}. The second challenge 
is an informed approach, the within-class-imbalance. Within-class-imbalance refers to identifying sparse or 
dense clusters of minority or majority instances. To tackle this challenge, different clustering based 
oversampling methods have been proposed to identify areas, where oversampling is most effective. These 
methods are segmenting the instances and then apply traditional oversampling methods specific to each segment. 

Cluster SMOTE utilizes the k-means cluster algorithm, to identify clusters with a specific threshold of 
minority instances, before applying SMOTE within these clusters \cite{CieslakCS06}. In addition, K-means 
and SMOTE applies a similar approach, but also considers the density within the clusters by assigning 
samples weights. A high sampling weight corresponds to a low density of minority samples and yields 
more generated samples \cite{Last2017}. Furthermore, multiple suggestions for the Hyperparameter selection 
are provided. DBSMOTE provides another approach to cluster based over-sampling, by applying the 
density-based DBSCAN algorithm to discover arbitrarily shaped clusters to create artificial data 
instances along the shortest path from each minority class instance to a pseudo-centroid of the 
cluster \cite{Bunkhumpornpat2011}. A-SUWO uses a hierarchical clustering approach and adaptively 
determines the size to oversample each sub-cluster using its classification complexity and cross-validation
 \cite{Nekooeimehr2015}. 

The Self Organizing Map Oversampling (SOMO) algorithm applies a self-organizing map to create a two-dimensional 
representation of the multidimensional input space and creates inter- and intra-cluster synthetic 
samples based on the underlying manifold structure \cite{Douzas2017B}. The algorithm transforms the 
input in a two-dimensional space of clusters and applies the SMOTE over-sampling algorithm to generate 
synthetic instances in the minority clusters and between neighboring minority clusters. The densities 
in and between the clusters are also taken into account by utilizing the average Euclidean distance. 
The algorithm stands out through its ability to not only oversample the identified clusters, but also 
between neighboring minority clusters since those areas are also safe and effective for data generation.  

Besides clustering based algorithms, further over-sampling algorithms evolved, that are based on 
ensemble methods like SMOTEBoost \cite{Chawla2003} and DataBoost-IM \cite{Guo2004}. The boosting 
process is used to identify hard to learn instances, which are used to separately generate synthetic 
examples for the majority and minority classes  \cite{Guo2004}.

\section{Motivation}

The section briefly outlines the challenges and shortcomings of competitive oversampling methods 
and clarifies the motivation for our proposed algorithm G-SOMO.

A classifiers performance is significantly influenced by the positioning of the instances in the 
dataset. As outlined in figure \ref{fig:Saez}, the instances of a class can be spread over the 
boundaries of the majority class, which complicates the classifiers task \cite{Tang}. This is a 
common challenge in real-world problems. The different types of instances can be seen as safe and 
unsafe instances, while the latter also differentiate between borderline and outlier instances \cite{Saez}.

\begin{figure}[H]
	\centering
	\includegraphics[width=10cm,height=7cm, keepaspectratio]{./resources/DifferentKindOfDataInstances.png}
	\captionbelow{Challenging data formations, inspired by \cite{Saez}.}
	\label{fig:Saez}
\end{figure}

Safe instances are located in the homogeneous regions and populated by the examples from one class only 
\cite{rodriguez2012hybrid}. They are clearly separated from examples of other classes. Most classifiers 
are able to correctly identify those instances \cite{Prati2004B}. Instances that are not clearly separated 
are considered unsafe ones like borderline instances and outliers \cite{Kubat2000}. Borderline instances 
are placed in the boundary region between classes, where instances of multiple classes overlap \cite{Saez}. 
Outliers are isolated distinct instances, also termed as noise. 

Learning algorithms are challenged when they are confronted with overlapping areas between classes, especially 
unsafe instances. Therefore, oversampling algorithms should be able to carefully consider in which areas to
 create artificial instances and which areas to ignore. Inefficiencies that were observed with SMOTE and similar
  algorithms are the generation of noisy instances penetrating the area of majority classes, as well as the 
  generation of nearly duplicated examples. Noisy examples can be created when applying k-nearest neighbor 
  approaches, since they can either choose a noisy instance as a starting point or select a noisy instance as 
  nearest neighbor, as seen in figure 3a and 3b. Similar or nearly duplicated instances are created while 
  generating new artificial instances within the borders of a minority cluster since they do not help to 
  strengthen the decision boundary and might lead to overfitting, as it can be seen in figure 3c. Figure 3d 
  outlines the case, in which the parameter k is too high and individuals from another cluster are selected.

\begin{figure}[H]
	\centering
	\includegraphics[width=16.5cm, height=14cm, keepaspectratio]{./resources/OversamplingChallenges2.png}
	\captionbelow{Shortcomings of SMOTE, inspired by \cite{Douzas2017B}}
\end{figure}

The Self-Organizing Map Oversampling (SOMO) algorithm improved the selection criteria of the minority class 
samples, which are used to generate synthetic examples. Through the synthetic data generation in and between
 neighboring minority clusters, as well as the consideration of the density, the algorithm also manages to 
 generate the synthetic instances in more productive areas of the data space \cite{Douzas2017B}. Therefore, 
 the generation of noisy examples is avoided. SOMO manages to identify efficient oversampling areas, the 
 generation of synthetic instances is then handled by the SMOTE algorithm, which still provides several 
 drawbacks. Challenges, as outlined in figure 3d, are prevented through the previous generation of clusters, 
 nevertheless, the SMOTE algorithm introduces a high correlation between the samples, since it only creates 
 synthetic instances that are on the line segment between two instances.

Geometric SMOTE (G-SMOTE) extends the linear interpolation mechanism by introducing a geometric region, in which 
the data generation process occurs \cite{Douzas2017}. Therefore, the algorithm extends the linear interpolation 
mechanism and provides a geometric region in which the data generation occurs. Figure 4 illustrates the idea 
behind the data generation based on a geometric region, instead of on the line segment. 

\begin{figure}[H]
	\centering
	\includegraphics[width=12cm, height=9.5cm, keepaspectratio]{./resources/GSMOTEHypersphere.png}
	\captionbelow{Synthetic data generation based on a geometric region, inspired by \cite{Douzas2017}}
\end{figure}

Besides utilizing a geometric region for the data generation, G-SMOTE also introduced the majority selection 
strategy, which successfully prevents the scenario shown in figure 3a, 3b and 3d. The majority selection 
strategy prevents the creation of synthetic instances between two minority instances if the distance to a 
majority instance is shorter. Instead, the synthetic minority instance is created between the initially selected
 minority instance and the majority instance that had a shorter distance, as it can be seen in figure 4. 
 G-SMOTE proved to significantly increase the performance of SMOTE. 

SOMO provides an efficient process to identify areas that should be populated with minority instances, but lacks 
the improvements provided by G-SMOTE. On the other hand, G-SMOTE exploits an intelligent and efficient approach
 to generate synthetic instances, but lacks the ability to identify attractive regions for synthetic instances.
  In this paper we propose a combination of both techniques, leveraging the benefits of both algorithms. 

\section{Proposed method}

The previous section outlined the inefficiencies of the SMOTE algorithm, as well as of SOMO and G-SMOTE. The proposed
 method G-SMOTE manages to tackle them, by utilizing the best characteristics of both algorithms. 

\begin{itemize} 
\item G-SOMO  manages to improve the selection criteria by leveraging Self-Organizing Maps.
\item The method considers the density, based on the average Euclidean distances, in and between clusters when 
	  creating artificial data instances.
\item G-SOMO creates synthetic instances within a geometric region, avoiding correlation between the instances.
\item Through the combined selection strategy, the creation of noisy examples is avoided. This should 
	  already be prevented through the preselection of minority clusters, but still provides an advantage, in case 
	  the parameters to cluster were not selected carefully.
\end{itemize}

The proposed method consists of three stages. Initially, a Self-Organizing Map (SOM) is applied to the normalized
 input data set. The SOM algorithm creates mappings of high dimensional data space into low-dimensional 
 space in such a way that the topological relations of the input patterns are preserved \cite{KOKUER2007}. 
 To train a SOM, the Euclidean distance between the input vector and all neural weights has to be calculated. 
 The Neuron that has the shortest distance to the input vector (the winner) is chosen and its weights are 
 slightly modified to the direction represented by the input vector. Afterward, the neighboring neurons are 
 taken and their weights are modified in the same direction \cite{Brocki2007}. Due to SOMs ability to preserve 
 topological relations from high dimensional input spaces, insights in the underlying data structure can be 
 retrieved by analyzing the (usually) two-dimensional output map. In the second stage of the algorithm, the 
 filtered clusters are defined, clusters where the minority class is dominating over the majority class. A weighted 
 approach is used to create synthetic instances in the filtered clusters and also between neighboring filtered 
 clusters. The last stage of the algorithm is marked by the creation of synthetic instances within the 
 identified formations. The synthetic data generation is based on a geometric region, that is formed between the 
 current minority instance and its selected neighbor. The shape of the geometric region varies, depending on its 
 hyperparameter. Synthetic instances are created within the geometric region, avoiding the creation on a line 
 segment as suggested by SMOTE. 

\subsection{G-SOMO Algorithm}

The G-SOMO algorithm is the efficient combination of the SOMO and G-SMOTE algorithm, introduced in 
\cite{Douzas2017B} and \cite{Douzas2017}. The complete G-SOMO algorithm is listed below.

\begin{algorithm}[H]
	\SetAlgoLined
	\caption{Pseudo code for G-SOMO implementation}
	
	
	\SetKwInOut{Input}{Input}
	\SetKwInOut{Output}{Output}
	
	\underline{G-SOMO algorithm}($args$): \\
	
	\Input{$S_{maj}$ (Set of majority class samples) \\
	$S_{min}$ (Set of minority class samples) \\
	$N_{intra}$ (Total intracluster number to be generated) \\
	$N_{inter}$ (Total intercluster number to be generated ) \\
	$filtered\_cluster_{ratio}$ (Treshold to identify a cluster as filtered) \\
	$inter\_intra\_cluster_{ratio}$ (Ratio of inter- and intracluster generated samples) \\
	$SOM_{parameters}$ (Parameters of SOM algorithm ) \\
	$\alpha_{trunc}$ and $\alpha_{def}$  (Parameters to form the geometric region) } 

	\textbf{Algorithm}
	\SetAlgoLined

	1. Normalize the data and train a SOM on the input data set, $S = S_{min} \cup S_{maj}$. \\
	2. Identify each node in the map as a cluster, where $cl$ describes all clusters \{1,2,...,$N_{Grid}^2$\}. $N_{Grid}$ is the dimension of the grid, one of the SOM parameters. \\
	3. Define the number of minority instances as $n_{+i}$ and the number of majority instances as $n_{-i}$ for each cluster $i \in cl$. \\
	4. Calculate the imbalance ratio $IR_{i}$ for each cluster $i \in cl$ as $IR_{i} = \dfrac{n_{+i} +1}{n_{-i} +1}$.\\
	5. Identify the filtered set $cl_{f}$ of cluster labels as $cl_{f} = \{i \in cl: IR_{i} > filtered\_cluster_{ratio}$. \\
	6. In each filtered cluster $i \in cl$ calculate the average Euclidean distance $dist_{i}$ across all pairs of positive class instances belonging to the cluster. \\ 
	7. Calculate the density factor for each filtered cluster as $den_{i} = \dfrac{n_{+i}}{dist_{i}^2}$. \\
	8. Identify the density factor of each filtered neighboring units $i$ and $j$ as $den_{ij} = den_{i} + den_{j}$ for each combination of $i, j \in cl_{f}$. \\
	9. Define the sampling weights\\
	\Indp 9.1 Calculate the intracluster weights as $w_{intra} = \dfrac{1 / den_{i}}{\sum_{i \in cl_{f}} 1/den_{i}}$ \\
		9.2 Calculate the intercluster weights as $w_{inter} = \dfrac{1 / den_{i,j}}{\sum_{i,j \in cl_{f}} 1/den_{i,j}}$ \\
	\Indm
	10. Define the number of artificial instances to be created \\
	\Indp 10.1 For each filtered cluster generate $w_{intra, i} \cdot N_{intra}$ artificial minority samples \\
		10.2 For each neighborhood combination of filtered clusters i,j generate $w_{intra, ij} \cdot N_{inter}$ artificial minority samples\\
	\Indm
	
	11. \For{ each filtered minority cluster formation and neighborhood formation }{
		11.1 Shuffle $S_{min_f}$, where $S_{min_f}$ are the minority instances of the current formation\\
	\Repeat{$N_{instances}$ of synthetic minority instances are created in the current formation}{
      		11.2.1 Select $x_{center}$ from $S_{min_f}$ \\
		11.2.2 Apply the combined selection strategy \\
		11.2.3 Generate a random point inside the unit-hypersphere centered at the origin of the input space\\
		11.2.4 Truncate the current unit hyper sphere \\
		11.2.5 Deform the unit hyper sphere to a hyper spheroid\\
		11.2.6 Rescale the created sample $x_{gen}$ \\
		11.2.7 Add the sample $x_{gen}$ to the set of generated instances in the current formation\\
    	}
	}	
	\Output{ $X$' (oversampled matrix of observations)} 	
\vspace{\baselineskip}
\end{algorithm}

\subsection{Explanation of G-SOMO}

The G-SOMO algorithm relies on filtered clusters, which are areas where a specific minority class 
dominates over the majority class. These regions can be considered as safe areas for the generation 
of minority samples. Outliers or noisy examples should belong to non-filtered clusters and are therefore 
ignored. The density and weights of each filtered cluster are calculated to create new instances in 
each filtered cluster (intracluster) and between filtered clusters that are neighbors on the topological 
output map of the SOM algorithm (intercluster). The geometric region used for the data generation increases
 the variety of the generated instances. 

\textbf{\underline{Parameters:}} The G-SOMO algorithm requires the following parameters. $S_{maj}$ 
representing the instances of the majority class, while $S_{min}$ represents the instances of the minority
 class. $N_{intra}$ is the absolute number of artificial instances that will be created within 
 clusters, $N_{inter}$ represents the absolute number of artificial instances that are created between 
 neighboring filtered clusters. $N_{intra}$ and $N_{inter}$ are obtained by the total number of artificial
  samples split by the $inter\_intra\_cluster_{ratio}$, which describes the distribution of the total 
  amount of artificial samples to be generated between the inter and intracluster process. The 
  $filtered\_cluster_{ratio}$ has to be defined between 0 and 1, describing the threshold to accept 
  a cluster as filtered. One notable SOM parameter is the size of the grid, which describes the 
  number of nodes and therefore clusters. Typical parameters to form the geometric region such 
  as $\alpha_{trunc}$ and $\alpha_{def}$ are required, their function will be explained in the 
  corresponding steps. 

\textbf{\underline{Step 1:}} The first step of the algorithm is to normalize the input data and apply 
the SOM algorithm with the provided parameter. The data is normalized that each feature has a mean 
of 0 and a variance of 1.  This step is necessary to assure that each feature is aligned on the same 
scale when assigning the best matching unit by utilizing the Euclidean distance during the training 
phase of the SOM algorithm. After normalizing the data, the correct size of the grid has to be chosen, 
which is a crucial parameter of the G-SOMO algorithm. The high dimensional input space will be 
transformed into a two-dimensional grid that consists of $N_{Grid}^2$ clusters, which are used to 
identify safe areas for the data generation process. The challenge hereby is to select a value 
allowing to discriminate between sparse and dense minority class areas.  A very small value will 
not be able to identify subclusters, the identified clusters will have a very large size of instances. 
Assuming a uniform distribution, one can set the threshold to  $\sqrt{|S_{min}|}$ to ensure that each 
cluster contains on average one minority class,  but the assumption of the distribution is unreliable 
for real-world problems. High values of $N_{Grid}$ will result in more smaller sized clusters, which 
might lead to filtered clusters in areas that we would usually like to ignore, because they are 
considered outliers. $\sqrt{|S_{maj}|}$ provides a reasonable upper bound for $N_{Grid}$. The optimal
 value for $N_{Grid}$ is dependent on the characteristics of the data set and can only be 
 approximated in an experimental approach.

\textbf{\underline{Step 2 - 5:}} The filtered clusters are identified by their Imbalance Ratio 
$IR_{i} = \dfrac{n_{+i} +1}{n_{-i} +1}$, where $n_{+i}$ is the number of instances belonging to the 
minority class and $n_{-i}$ are the instances of the majority class. Clusters having a higher ratio 
than the given $filtered\_cluster_{ratio}$ threshold are considered to be filtered clusters of the 
current class. The parameter $filtered\_cluster_{ratio}$ is usually set to 0.5. The G-SOMO extension 
will not create any artificial instances if no filtered clusters are identified. In this case, the 
minority instances are sparse and the distinction between noisy and informative instances is not 
possible. One possibility hereby is to increase the parameter $N_{Grid}$ to create more clusters. 

\textbf{\underline{Step 6 - 8:}} For each identified filtered cluster, the average Euclidean distance 
is calculated between the minority instances. Using the average Euclidean distance, we assign a 
density factor to each filtered cluster $den_{i} = \dfrac{n_{+i}}{dist_{i}^2}$. This measure provides 
information about the distribution of the instances within each cluster and will be helpful in further 
stages to assign the correct amount of artificial samples to each cluster. For each topological 
neighboring combination of filtered clusters, the density is defined as the sum of both individual 
density factors. Filtered clusters that are not topological neighbors, or have no topological neighbors 
are excluded from the process. Figure \ref{fig:Somo_Overview}a illustrates the SOM Grid with all identified 
filtered clusters, figure \ref{fig:Somo_Overview}b outlines the topological neighbor structures. In the 
next step, we calculate the relative weight of each filtered cluster and each filtered cluster neighborhood 
relationship by using its density. The weight determines the number of synthetic samples that will be 
created in each filtered cluster and between filtered clusters. 

\begin{figure}[H]
	\centering
	\includegraphics[width=18cm,height=13cm, keepaspectratio]{./resources/SOMO_process2.png}
	\captionbelow{Overview of identified clusters and neighboring relations, inspired by \cite{Douzas2017B}}
	\label{fig:Somo_Overview}
\end{figure}

\textbf{\underline{Step 9 - 10:}} The following steps determine an efficient number of synthetic 
instances for each filtered cluster and each neighboring relation of filtered clusters. $N_{intra}$ and $N_{inter}$ 
provide the guideline of the total synthetic instances to be created in the inter and intracluster process.  
Utilizing the density information, a relative weight is calculated for each filtered cluster 
as $w_{intra} = \dfrac{1 / den_{i}}{\sum_{i \in cl_{f}} 1/den_{i}}$, where 1 divided by the density 
of the current cluster is divided by 1 through the sum of all densities of filtered clusters. 
Hereby, we obtain a relative weight of each filtered cluster based on the cluster size and the density 
of each cluster. $N_{intra}$ times the weight for each specific filtered cluster results in the amount of 
artificial data that is generated in each filtered cluster. Afterwards the weights of the neighboring 
clusters are calculated as $w_{inter} = \dfrac{1 / den_{i,j}}{\sum_{i,j \in cl_{f}} 1/den_{i,j}}$ in a 
similar manner as the intra weight calculation. Once each neighborhood relation has a relative weight 
assigned, $N_{inter}$ times the relative weight results in the number of synthetic instances that are 
created in each filtered cluster neighboring relation. In case there are no neighborhood relations for 
a class all the artificial samples are created through the intracluster process. 

\textbf{\underline{Step 11 - 11.2.2:}}  In previous steps, we managed to identify the corresponding 
number of required synthetic instances for each filtered cluster formation and each neighboring relation 
formation. The following process is applied to each formation individually. The set of minority instances 
of the current formation is shuffled, minority instances are repetitively selected, each minority 
instance can be selected multiple times if the number of required synthetic instances is higher than the 
number of minority instances of the formation. The current selected minority sample of $N_{instances}$ is 
named  $x_{center}$. Based on $x_{center}$ the combined majority selection strategy is applied. The combined 
selection strategy identifies $x_{surface}$, the final neighbor of $x_{center}$ that is used for over-sampling 
by applying the minority and majority selection strategy, both based on a k-nearest neighbor approach. The 
minority selection strategy selects a neighbor instance $x_{min}$ within the k-nearest minority instances of 
$x_{center}$, similar to SMOTE. The majority selection strategy selects $x_{maj}$, a neighboring majority 
instance that has been identified with the identical approach applied to neighboring majority instances. 
$x_{surface}$ is defined as the instance $x_{min}$ or $x_{maj}$ that is closer to $x_{center}$, based on 
its distance. The distance between $x_{center}$ and $x_{surface}$ is $R$. Figure \ref{fig:Combined}a 
outlines the result of the majority and minority selection strategy, each strategy proposes one nearest 
neighbor of its class, the combined selection strategy in figure \ref{fig:Combined}b selects the 
nearest sample of both selection strategies as $x_{surface}$.

\begin{figure}[H]
	\centering
	\includegraphics[width=18cm,height=7cm, ]{./resources/min_maj_comb_selection2.png}
	\captionbelow{Procedure of the combined selection strategy, inspired by \cite{Douzas2017}}
	\label{fig:Combined}
\end{figure}

\textbf{\underline{Step 11.2.3 - End:}} To identify the most suitable geometric shape for the data 
generation process between $x_{center}$ and $x_{surface}$, a unit hypersphere is established around 
$x_{center}$, as it can be seen in figure \ref{fig:Hypersphere}a . This hypersphere is going to be modified 
within the next steps. Initially, a random point $e_{sphere}$ is created on the edge of the unit 
hypersphere, centered around $x_{center}$. Subsequent, $e_{sphere}$ is transformed to $x_{gen}$ a 
random point along the line segment, uniformly distributed. Step 11.2.4 applies a truncation, a partition 
of the hypersphere, as illustrated in \ref{fig:Hypersphere}b. The parameter $\alpha_{trunc}$ determines 
the degree of the truncation, as seen below. The truncated area is orthogonal to the unit vector $e_{//}$, 
where $e_{//}$ defines the direction between $x_{center}$ and $x_{surface}$. If $\alpha_{trunc}$ is bigger 
than zero, the area that does not include $x_{gen}$ is truncated. In case that $x_{gen}$ is within the 
truncated area, the point is reflected on the opposite side of the hypersphere.
\\

\begin{figure}[H]
	\centering
	\includegraphics[width=15cm,height=10cm, keepaspectratio]{./resources/trunc_def3.png}
	\captionbelow{Constructing the geometric region, based on the unit hyper sphere, truncation and 
	deformation, inspired by \cite{Douzas2017B}}
	\label{fig:Hypersphere}
\end{figure}

Step 11.2.5 deforms the truncated hypersphere to a hyper spheroid, as it can be seen in \ref{fig:Hypersphere}c 
The parameter $\alpha_{def}$ controls the degree of the deformation. The point $x_{gen}$ is moved to the same 
extent in the orthogonal direction to the unit vector $e_{//}$. The previous steps of truncation and 
deformation modify the initially uniform distribution of the new point $x_{gen}$. Due to the truncation 
and deformation, we take better account of the characteristics of $x_{center}$ and $x_{surface}$, 
while still having a higher variability than methods based on line segments, such as SMOTE. In a final 
step we rescale the point $x_{gen}$ based on $x_{center}$ and the distance to $x_{surface}$, $R$. The 
rescaled outcome of the forming process can be seen below in figure \ref{fig:GeometricConstruction}. 
$X_{gen}$ is created within a hypersphere, that was truncated and deformed to a hyper spheroid.

\begin{figure}[H]
	\centering
	\includegraphics[width=12cm,height=9cm, keepaspectratio]{./resources/GSMOTE_result.png}
	\captionbelow{Rescaled result of geometric construction, inspired by \cite{Douzas2017B}}
	\label{fig:GeometricConstruction}
\end{figure}

The process is repeated in each formation until $N_{instances}$ are added to each formation. This results 
in the total number of synthetic instances, the sum of $N_{intra}$ and $N_{inter}$. The final output of 
the G-SOMO algorithm is the oversampled matrix $X$', consisting of the input data set and the added 
synthetic minority instances.

\section{Research methodology}

The objective of a sampling method is the improvement of the classification results. This implies that 
validation methods have to be chosen to provide reliable and comprehensive results. The correct metrics 
have to be selected to not create misleading improvements as outlined in the next section. To receive 
comparable results, all oversampling methods need to be trained with the same classifier algorithms 
and are compared to the performance of the classifier algorithm on the same data set without oversampling.

Selecting a reliable validation technique is a common challenge to assess the generalizability 
of a classifier. Oversampling methods can tend to encourage the process of $overfitting$. Overfitting 
implies that the classifier learned the data structure too well and lost its ability to generalize on 
unseen data. A popular approach to validate the performance of a classifier is to split the data set 
into two subsets, one used for the training phase of the algorithm and the other one remains unseen 
during the training phase and is only used for validation, the test set. As a matter of fact, this 
validation method can even increase the process of overfitting, if the dataset is not split randomly. 
Certain characteristics of a feature might only occur in the test set and might not appear in the 
train set and vice versa. A more sophisticated method is k-fold Cross-Validation. The data set is split 
into k different subsets (also called folds). K-1 subsets are used to train the classifier and the 
last fold remains unseen to validate. The process is applied in an iterative manner, such that each fold 
is used for validation once. The validation results of each model are averaged afterwards. The process 
can be repeated multiple times to avoid bias due to random grouping \cite{Japkowicz2013}. An extension 
to k-fold provides the stratified k-fold method. In stratified cross-validation, the random folds are 
chosen such that the class distribution in each fold is maximally similar to the class distribution in 
the whole dataset \cite{Vanwinckelen2015}. The method successfully avoids the problem of an unequally 
split dataset, but faces problems if a minority class is highly underrepresented and does not have 
enough samples to split them equally in k folds.

\subsection{Metrics}

The model evaluation is based on different evaluation metrics. Traditional metrics, such as accuracy, 
show a strong bias towards the majority class and are recommended to avoid on imbalanced datasets. The 
overall accuracy would seem to be precise, even though the model might not perform well on the minority 
class. To retrieve more accurate insights on a models performance, one should reflect upon the confusion 
matrix, outlined below: 

\begin{center}
\begin{tabular}{ c|c|c }
  & \textbf{Predicted as Positive} & \textbf{Predicted as Negative} \\  [1ex] 
  \hline
 \textbf{Actual Positive} & True Positive (TP) & False Negative (FN) \\  [1ex] 
 \textbf{Actual Negative} & False Positive (FP) & True Negatives (TN) \\ [1ex] 
\end{tabular}
\end{center}

Based upon the confusion matrix additional metrics, such as Precision and Recall \cite{Dalianis2018} were introduced: 

\begin{center} $ Recall = \frac{TP}{TP + FN} \qquad Precision =  \frac{TP}{TP + FP}$\end{center}

The evaluation of our experiments is conducted with the following metrics, that are based on the previously 
introduced fundamentals. These metrics proved to be reliable for imbalanced learning problems since they 
focus on both classes equally independent from their ratio.

\begin{itemize}
  \item F1 Measure (F1) \\
  \\
  The F-measure is described as the harmonic mean between precision and recall, assuming that both metrics 
  are of equal importance \cite{Guo2018}. The F1 Score is defined as follows:
  
  \begin{center} $F1$  $Measure= 2 * \frac{Precision * Recall}{Precision + Recall}$ \end{center}
    
  \item Geometric Mean Score (G-Mean)
  
  The geometric mean score, as the name implies, is defined as the geometric mean between True Positive Rate 
  (TPR) and True Negative Rate (TNR). The g-measure ranges between 0 and 1 and considers the TPR and TNR 
  with equal importance. The metric is defined as follows:
  
   \begin{center} $G-Mean=  \sqrt{\frac{TP}{TP + FN} * \frac{TN}{TN + FP}}$ \end{center}
  
  \item Area Under The Curve Receiver Operating Characteristics (AUC - ROC)
  
  The ROC Curve is created by plotting the TPR against the FPR \cite{Hand2009}. The AUC score is the 
  area under that curve. The better the model distinguishes the majority and minority class the better 
  the final score. 

\end{itemize}

subsection{Oversampling Methods}

The following section provides an overview of the oversampling methods used as a benchmark to validate 
the effectiveness of G-SOMO. The baseline oversamplers are introduced in the second section of the paper 
and will therefore not be presented in detail. The oversamplers used for our benchmark results are 
popular techniques for binary classification tasks.

The optimal imbalance ratio to oversample is not known and might vary between datasets \cite{Provost2018}. 
Usually, oversamplers are utilized to generate as much minority instances as required to equal the class d
istribution. Within our experimental framework, G-SOMO is the only oversampling method that uses an 
informative approach to identify effective areas to oversample. In case that no area is identified, 
the algorithm will not create any synthetic instances. The oversamplers used in our experimental 
framework are:

\begin{itemize}
  \item No Oversampling
  \item Random Oversampling 
  \item SMOTE 
    \begin{itemize}
 \item knn = \{3,5\}
 \end{itemize}
  \item G-SOMO 
  \begin{itemize}
 \item knn = \{3,5\},
 \item truncation\_factor = \{-1.0, 0.0, 0.25, 1.0\},
 \item deformation\_factor = \{0.0, 0.5, 1.0\},
 \item Grid\_size = \{0.2, 0.5\}
 \end{itemize}
\end{itemize}

While the parameters were already introduced during the explanation of the G-SOMO algorithm, it is 
worth to outline the Grid\_size parameter. Hereby, we determine the number of clusters proportional 
by the number of input samples multiplied with the relative Grid\_size parameter.

\subsection{Classifiers}

Different classifier algorithms are chosen to evaluate the performance of the oversampling methods. 
It is crucial to ensure that the obtained results are generalizable to different classifiers and not 
only to specific ones, due to the different algorithms characteristics. To reduce biased results only
 algorithms with a low number of hyperparameter are chosen. The chosen classifiers are Logistic 
 Regression (LR), K-Nearest Neighbors (KNN), Decision Tree (DT) and Gradient Boosting Classifier (GBC).

The Logistic Regression is used to model the outcomes of a categorical dependent variable and 
predicts the probabilities for the different possible outcomes based on several independent 
variables. Fitting a linear model is an optimization problem which can be solved using simple 
optimizers which require no hyperparameters to be set \cite{McCullagh1984}.

KNN is a popular algorithm, in which a new instance is classified into the class with the most 
members present among the k nearest neighbors \cite{Suguna2010}. The hyperparameter k describes 
the number of neighbors considered for the classification.

Decision Trees create their classification decisions based on a tree structure that was 
obtained during the learning process. The final result is a tree with decision nodes and 
leaf nodes, where a decision node has two or more branches and leaf nodes represent the decision.

Gradient boosting is a technique that creates an ensemble of underlying weak learners to 
perform better than random guessing. By combining these weak learners based on a weighted 
majority vote, a committee classifier dramatically reduces the training and testing error 
rates  \cite{Huang2007}. The number of trees to create is a hyperparameter of the algorithm.

All classifiers are trained with different values of hyperparameters, besides the LR, which does 
not require any hyperparameter to be set. The following list provides an overview of the 
classifiers used with its hyperparameters:

\begin{itemize}
  \item LR
  \item KNN (k = \{3,5\})
  \item DT (max\_depth = \{3,6\})
  \item GBC (max\_depth = \{3,6\})
\end{itemize}

\subsection{Datasets}

To ensure meaningful and significant insights we evaluated our models on a total of 69 datasets. 
These datasets consist of commonly used datasets mainly from UCI Machine Learning repository. 
In order to reach this high number of datasets, we randomly undersampled the minority class of 
some datasets to increase the Imbalance Ratio. Using this approach, we could create additional 
and more challenging datasets. Table 1 provides an overview of our datasets, the number of 
features, instances and their Imbalance Ratio.

\begin{table} [H]
   \scriptsize
   \centering

\begin{tabular}{ |p{4cm}||p{2.2cm}|p{2.2cm}|p{2.2cm}|p{2.2cm}|p{2.2cm}|  }  
 \hline
 \multicolumn{6}{|c|}{Data Sets} \\
 \hline
 Datasets & \# Features & \# Instances & \# Minority Instances & \# Majority Instances & Imbalance Ratio \\
 \hline
 
 
BREAST TISSUE&	9&	106&	36&	70&	1.94\\
BREAST TISSUE (2)&	9&	88&	18&	70&	3.89\\
DERMATOLOGY&	34&	358&	20&	338&	16.9\\
ECOLI&	7&	336&	52&	284&	5.46\\
ECOLI (2)&	7&	310&	26&	284&	10.92\\
ECOLI (3)&	7&	301&	17&	284&	16.71\\
EUCALYPTUS&	8&	642&	98&	544&	5.55\\
EUCALYPTUS (2)&	8&	593&	49&	544&	11.1\\
EUCALYPTUS (3)&	8&	576&	32&	544&	17.0\\
GLASS&	9&	214&	70&	144&	2.06\\
GLASS (2)&	9&	179&	35&	144&	4.11\\
GLASS (3)&	9&	167&	23&	144&	6.26\\
HABERMAN&	3&	306&	81&	225&	2.78\\
HABERMAN (2)&	3&	265&	40&	225&	5.62\\
HABERMAN (3)&	3&	252&	27&	225&	8.33\\
HEART&	13&	270&	120&	150&	1.25\\
HEART (2)&	13&	210&	60&	150&	2.5\\
HEART (3)&	13&	190&	40&	150&	3.75\\
IRIS&	4&	150&	50&	100&	2.0\\
IRIS (2)&	4&	125&	25&	100&	4.0\\
IRIS (3)&	4&	116&	16&	100&	6.25\\
LED&	7&	443&	37&	406&	10.97\\
LED (2)&	7&	424&	18&	406&	22.56\\
LIBRAS&	90&	360&	24&	336&	14.0\\
LIVER&	6&	345&	145&	200&	1.38\\
LIVER (2)&	6&	272&	72&	200&	2.78\\
LIVER (3)&	6&	248&	48&	200&	4.17\\
MANDELON 1&	20&	4000&	142&	3858&	27.17\\
MANDELON 1 (2)&	20&	3929&	71&	3858&	54.34\\
MANDELON 1 (3)&	20&	3905&	47&	3858&	82.09\\
MANDELON 2&	200&	3000&	105&	2895&	27.57\\
MANDELON 2 (2)&	200&	2947&	52&	2895&	55.67\\
MANDELON 2 (3)&	200&	2930&	35&	2895&	82.71\\
NEW THYROID 1&	5&	215&	35&	180&	5.14\\
NEW THYROID 1 (2)&	5&	197&	17&	180&	10.59\\
NEW THYROID 2&	5&	215&	35&	180&	5.14\\
NEW THYROID 2 (2)&	5&	197&	17&	180&	10.59\\
PAGE BLOCKS 0&	10&	5472&	559&	4913&	8.79\\
PAGE BLOCKS 0 (2)&	10&	5192&	279&	4913&	17.61\\
PAGE BLOCKS 0 (3)&	10&	5099&	186&	4913&	26.41\\
PAGE BLOCKS 1 3&	10&	472&	28&	444&	15.86\\
PIMA&	8&	769&	268&	501&	1.87\\
PIMA (2)&	8&	635&	134&	501&	3.74\\
PIMA (3)&	8&	590&	89&	501&	5.63\\
SEGMENTATION&	16&	2310&	330&	1980&	6.0\\
SEGMENTATION (2)&	16&	2145&	165&	1980&	12.0\\
SEGMENTATION (3)&	16&	2090&	110&	1980&	18.0\\
VEHICLE&	18&	846&	199&	647&	3.25\\
VEHICLE (2)&	18&	746&	99&	647&	6.54\\
VEHICLE (3)&	18&	713&	66&	647&	9.8\\
VOWEL&	13&	988&	90&	898&	9.98\\
VOWEL (2)&	13&	943&	45&	898&	19.96\\
VOWEL (3)&	13&	928&	30&	898&	29.93\\
WINE&	13&	178&	71&	107&	1.51\\
WINE (2)&	13&	142&	35&	107&	3.06\\
WINE (3)&	13&	130&	23&	107&	4.65\\
YEAST 1&	8&	1484&	429&	1055&	2.46\\
YEAST 1 (2)&	8&	1269&	214&	1055&	4.93\\
YEAST 1 (3)&	8&	1198&	143&	1055&	7.38\\
YEAST 3&	8&	1484&	163&	1321&	8.1\\
YEAST 3 (2)&	8&	1402&	81&	1321&	16.31\\
YEAST 3 (3)&	8&	1375&	54&	1321&	24.46\\
YEAST 4&	8&	1484&	51&	1433&	28.1\\
YEAST 4 (2)&	8&	1458&	25&	1433&	57.32\\
YEAST 4 (3)&	8&	1450&	17&	1433&	84.29\\
YEAST 5&	8&	1484&	44&	1440&	32.73\\
YEAST 5 (2)&	8&	1462&	22&	1440&	65.45\\
YEAST 6&	8&	1484&	35&	1449&	41.4\\
YEAST 6 (2)&	8&	1466&	17&	1449&	85.24\\

 \hline
\end{tabular}

\caption{Overview of all 69 datasets.}
   \label{tab:test}
\end{table} 


\subsection{Experimental framework}

Our experimental framework calculates the results of each oversampling method in combination 
with each classifier on each dataset based on each hyperparameter of the methods. The combination 
of all dependencies is named Grid Search, a search for the best results among all possible 
combinations. The result of each combination is obtained by the 5-fold cross-validation, 
repeated for 5 times, resulting in 25 models for each combination. Subsequently, the process 
is repeated for each metric. 

A ranking score is applied to compare the performance of the oversampling methods, aggregated 
over all datasets. The ranking score for the best performing method is 1, the worst performing 
methods is 4. The Friedman test is applied to the ranking results. The Friedman test is used 
to test differences between groups, assuming our target variable is categoric. The null hypothesis 
used in our tests states whether the classifiers have a similar performance across the 
oversampling methods and evaluation metrics when they are compared to their mean ranking.

The implementation of the classifiers and oversamplers are based on the python libraries Scikit-Learn 
\cite{Pedregosa2012} and Imbalanced-Learn \cite{Lemaitre2016}. The G-SOMO extension is implemented in 
Python, the code is available upon request.

\section{Experimental results}

For the purpose of a clear distinguishing between the oversampling methods, we introduced a ranking 
score, ranging from one to four. More precisely, one represents the best performance, four the worst. 
In order to accumulate the retrieved insights, we averaged repetitions on each hyperparameter and 
on each dataset, as well as for each of the five repetitions during the cross-validation. The table 
below illustrates the mean rank of each oversampler on each metric and each classifier. 

\begin{table} [h]
   \scriptsize
   \centering

\begin{tabular}{ |p{1.4cm}||p{2cm}|p{2cm}|p{2cm}|p{2cm}|p{2cm}|  }  
 \hline
 \multicolumn{6}{|c|}{Experimental Results} \\
 \hline
 Classifier & Metrics & No Oversampling & Random Oversampling & SMOTE & G-SOMO \\
 \hline
 
LR&	ROC AUC&	3.014&	2.731&	2.536&	1.717\\
LR&	F1&	3.347&	2.942&	2.355&	1.355\\
LR&	G-SCORE&	3.695&	2.275&	2.021&	2.007\\
KNN&	ROC AUC&	3.079&	3.420&	2.144&	1.355\\
KNN&	F1&	3.398&	2.753&	2.5&	1.347\\
KNN&	G-SCORE&	3.782&	2.420&	1.789&	2.007\\
DT&	ROC AUC&	3.318&	2.898&	2.311&	1.471\\
DT&	F1&	3.094&	2.971&	2.594&	1.340\\
DT&	G-SCORE&	3.768&	2.521&	1.949&	1.760\\
GBC&	ROC AUC&	3.007&	3.275&	2.565&	1.152\\
GBC&	F1&	3.297&	2.920&	2.442&	1.340\\
GBC&	G-SCORE&	3.586&	2.731&	2.050&	1.630\\

 \hline
\end{tabular}

\caption{Overview of results, Mean Ranking for each Classifier and Metric.}
   \label{tab:test}
\end{table} 

Taking a closer look at the averaged mean scores in table 2, one can determine certain characteristics. 
As expected, not using any oversampling method provides the worst results among all classifiers and 
metrics. Random oversampling performs better on the mean ranking, while SMOTE is superior to both of 
them. G-SOMO consistently outperformed other oversampling methods. We may observe that G-SOMO provides 
exceptional good mean ranking scores in combination with the metrics 'ROC AUC' and 'F1'. In one case 
SMOTE achieved a better mean score while validation with a Decision Tree on the Geometric-Mean Score. 
In total we can see that G-SOMO consistently outperformed other oversampling methods, the results in 
combination with Gradient Boosting Classifier can be emphasized. 

Based on the insights from table 2 we can conclude, that G-SOMO successfully outperforms other 
oversampling methods, proven in combination with various Metrics and Classifiers, averaged over 
all datasets. Figure \ref{fig:MeanRanking} underlines the dominance of G-SOMO, the graph provides 
the aggregated results or each classifier, hereby we can see that throughout all experiments 
G-SOMO demonstrates the best performance. Moreover, it highlights the previously mentioned ranking 
between all oversampling methods. 

\begin{figure}[H]
	\centering
	\includegraphics[width=15cm,height=12cm, keepaspectratio]{./resources/Mean_Ranking_Plot.png}
	\captionbelow{Graphical overview of Results, Mean Ranking for each Classifier}
	\label{fig:MeanRanking}
\end{figure}

Besides the introduced mean ranking, we also analyze the relative differences in the metric scores in 
depth since the mean ranking might conceal high variation in specific repetitions. To avoid this scenario 
we observe the metrics scores overall results and aggregate them across all datasets and repetitions. 
To provide a direct comparison between the performance of the oversampling methods, the results of 
G-SOMO are subtracted of each oversampling methods. The positive or negative difference implies whether 
the algorithm performed better or worse than the baseline G-SOMO. The differences are summarized in figure 
\ref{fig:AvgScore}. The differences of the oversampling methods are partitioned across each classifier 
and each metric. The colored gradation illustrates the relative difference between the specific 
oversampling method and the performance of G-SOMO.

The analysis of figure \ref{fig:AvgScore} reaffirms the previous insights, the majority of results are 
negative and therefore imply worse results than G-SOMO. The poor performance of no oversampling can be 
noted, the performance difference is up to 33\%, random oversampling performs slightly better, SMOTE 
is the most challenging method. As previously recognized, G-SOMO performs exceptionally with the 
'ROC AUC' and 'F1' measure. SMOTE manages to perform slightly better in the combination of the 
Geometric-Mean Score and KNN and LR classifier. Random Oversampling also performs better than 
G-SOMO in the combination of the Geometric Score and LR. Besides these exceptional cases, G-SOMO 
constantly outperforms all oversampling methods. The insights retrieved by the performance comparison 
is consistent with the insights obtained from the mean rankings. 

\begin{figure}[H]
	\centering
	\includegraphics[width=17cm,height=12cm, keepaspectratio]{./resources/ResultsComparedToG-SOMO.png}
	\captionbelow{Average Score of Oversampling methods compared to G-SOMO performance}
	\label{fig:AvgScore}
\end{figure}

To prove the significance of our experiments, a Friedman test is applied to our results. The exact results 
are shown in table 3. Hereby, we can obtain that the null hypothesis is rejected by far for a significance 
level of a=0.05 for all classifiers and metrics. Therefore, we can assume that our obtained results are 
statistically significant. 

\begin{table} [h]
   \scriptsize
   \centering

\begin{tabular}{ |p{2cm}||p{2cm}|p{4cm}|p{2cm}|  }  
 \hline
 \multicolumn{4}{|c|}{Friedman Test} \\
 \hline
Classifier & Metric & p-value & Significance \\
 \hline
 
LR&	ROC AUC&	2.978464111147552e-09&	True\\
LR&	F1&	1.6016121978014913e-21&	True\\
LR&	G-SCORE&	1.1962255259807236e-18&	True\\
KNN&	ROC AUC&	1.5777728619737928e-24&	True\\
KNN&	F1&	1.2633506344258716e-20&	True\\
KNN&	G-SCORE&	1.8594416543208717e-22&	True\\
DT&	ROC AUC&	4.4812704879505726e-18&	True\\
DT&	F1&	3.587895642878212e-18&	True\\
DT&	G-SCORE&	4.37730518573295e-23&	True\\
GBC&	ROC AUC&	3.6730748169610812e-25&	True\\
GBC&	F1&	1.2458033635808544e-20&	True\\
GBC&	G-SCORE&	5.957134339289162e-21&	True\\

 \hline
\end{tabular}

\caption{Results of Friedman Test.}
   \label{tab:test}
\end{table} 

In general we can observe that the informed geometric oversampling approach of G-SOMO has proven 
to be a successful new approach to handle imbalanced datasets. Our experiments prove the need for 
oversampling methods, using no oversampling methods consistently provided the worst results. We 
have found that Random Oversampling performs better than no oversampling, and SMOTE ranks second 
among our comparison in total. Based on our obtained insights, we determine that G-SOMO clearly 
outperformed other methods, often with great differences towards the other oversamplers. However, 
G-SOMO also requires a more intensive hyperparameter search, which in turn requires more 
computational resources. 

\section{Conclusion}

In this paper, we proposed a new oversampling algorithm G-SOMO. The algorithm observes the characteristics
 of the multidimensional data input while grouping the input data to identify filtered clusters, where 
 minority instances dominate. Within these filtered clusters, as well as between neighboring filtered 
 clusters we create synthetic instances. During the creation of synthetic instances, we apply the combined 
 selection strategy, that also takes near majority instances into account. The synthetic instances are 
 created in a safe hyper-spheroid. 

G-SOMO was evaluated on 69 different datasets and compared to popular oversampling methods. We validated 
the performance with different metrics, such as the F1 Score, the G-Score and the AUC-ROC. In order to 
avoid overfitting to a specific classifier, we chose multiple ones that differ in their characteristics. 
Each experiment is repeated 5 times with a 5-fold cross-validation. We proved the statistical significance 
of our results by a Friedman test.

In particular, our empirical results highlight the need for oversampling algorithms. Utilizing no oversampling 
method produced the worst results in our mean ranking, simple methods like random oversampling already 
increased the performance. The popular method SMOTE performed better than the previous ones, whereas G-SOMO 
dominated the ranking and outperformed other oversamplers. 

We are confident that G-SOMO is a new appealing approach for researcher and practitioners working with 
imbalanced datasets. 

Future work will focus on a more efficient estimation of the algorithms hyperparameter. During our experiments 
we noticed the issue to be challenging, through additional research one might identify generalizable 
characteristics for the parameters based on the datasets characteristics. Additional improvements can be 
expected through further research. Alternatively, the behavior of the algorithm on imbalanced datasets with 
multiple target classes is worth to explore, little research about oversampling on multiclass classification 
problems is made. 

\bibliography{references}
\bibliographystyle{apalike}

\end{document}