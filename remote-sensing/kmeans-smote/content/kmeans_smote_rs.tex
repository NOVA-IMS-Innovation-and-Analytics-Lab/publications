\documentclass[parskip=full]{scrartcl}

\pdfoutput=1

\title{Improving Imbalanced Learning in Land Cover Classification \\ 
	\LARGE{A Heuristic Oversampling Method Based on K-Means and SMOTE}}
\author{
	Georgios Douzas\(^{1}\), Fernando Bacao\(^{1*}\), Joao Fonseca\(^{1}\)
	\\
	\small{\(^{1}\)NOVA Information Management School, Universidade Nova de Lisboa}
	\\
	\small{*Corresponding Author}
	\\
	\\
	\small{Postal Address: NOVA Information Management School, Campus de Campolide, 1070-312 Lisboa, Portugal}
	\\
	\small{Telephone: +351 21 382 8610}
}

\usepackage{breakcites}
\usepackage{float}
\usepackage{graphicx}
\usepackage{subcaption}
\usepackage{geometry}
\geometry{
	a4paper,
	left=18mm,
	right=18mm,
	top=8mm,
}
\usepackage{amsmath}
\newcommand{\inlineeqnum}{\refstepcounter{equation}~~\mbox{(\theequation)}}
\usepackage{enumitem}
\usepackage[ruled,vlined]{algorithm2e}
\usepackage{booktabs}
\usepackage{pgfplotstable}
\pgfplotsset{compat=1.14}
\usepackage{longtable}
\usepackage{tabu}
\usepackage{hyperref}
\date{}

\begin{document}

\maketitle

\begin{abstract}

    Land cover maps are an important resource to make informed policy,
    development, planning and resource management decisions. The primary
    challenge for the development of accurate, timely and automated Land
    Use/Land Cover maps are technical skills. Specifically, remotely sensed data
    is often imbalanced, where the number of samples of a few classes is
    significantly greater the number of samples of the remaining classes. This
    asymmetric class distribution, impacts negatively the performance of
    classifiers and adds a new source of inaccuracy to the production of these
    maps. In this paper, we address this problem, known as the imbalanced
    learning problem, by using K-Means SMOTE, a recently proposed oversampling
    method. K-Means SMOTE is an oversampling algorithm that attempts to improve
    the quality of newly created artificial data by avoiding the generation of
    noisy data and effectively overcome data imbalance. The performance of
    K-Means SMOTE is compared to other popular oversampling methods using seven
    well known datasets and a variety of classifiers and evaluation metrics. The
    results show that the proposed method consistently outperforms the remaining
    oversamplers and produces higher quality land cover classifications.

\end{abstract}

\section{Introduction}

% context

The increasing amount of remote sensing missions granted the access to dense
time series (TS) data at a global level and provides up-to-date, accurate land
cover information \cite{Drusch2012}. This information is often materialized
through Land Use/Land Cover (LULC) maps, which constitute an essential asset for
various purposes, such as land cover change detection, urban planning,
environmental monitoring and natural hazard assessment \cite{Khatami2016}.
However, accurate and updated LULC maps are still a challenge within the remote
sensing community \cite{Wulder2018}, while their production follows two main
approaches: Photo-interpreted by the human eye, or Automatic mapping using
remotely sensed data and classification algorithms.

Although photo-interpreted LULC maps rely on human interaction and can be more
reliable, they are not without its drawbacks: they are not frequently updated,
their production is time and resource consuming, they are not suitable for
operational mapping over large areas and are prone to overlook rare or
small-area classes, due to factors such as the minimum mapping unit being used.
Similarly, machine-learning (ML) approaches face different challenges:
\begin{enumerate}
	\item Mislabelled LULC patches. As mentioned, the usage of photo-interpreted
	      training data poses a threat to the quality of any LULC map produced
	      with this strategy, since factors such as the minimum mapping unit
	      tend to cause the overlooking of small-area LULC patches and generates
	      noisy training data that may reduce the prediction accuracy of a
	      classifier \cite{Pelletier2017}.
	\item High-dimensional datasets. Multi-spectral TS composites are
	      high-dimensional, which increases the complexity of the problem and
	      creates a strain on computational power \cite{Stromann2020}.
	\item Class separability. The production of an accurate LULC map can be
	      hindered by the existence of classes with similar spectral signatures,
	      making these classes difficult to distinguish
	      \cite{Alonso-Sarria2019}.
	\item Existence of rare land cover classes. Due to the varying levels of
	      area coverage for each class, using a purely random sampling strategy
	      will amount to a dataset with a roughly proportional class
	      distribution as the one on the landscape. On the other hand, the
	      acquisition of training datasets containing balanced class frequencies
	      is often unfeasible. This causes an asymmetry in class distribution,
	      where some classes are frequent in the training dataset, while others
	      have little expression \cite{Wang2019, Feng2019}.
\end{enumerate}

% problem definition

The latter challenge is known as the imbalanced learning problem
\cite{Chawla2004}. It is defined as a skewed distribution of observations found
in a dataset among classes in both binary and multi-class problems
\cite{Abdi2016}. This asymmetry in class distribution negatively impacts the
performance of classifiers, especially in multi-class problems. During the
learning phase, classifiers are optimized to maximize an objective function,
with overall accuracy being the most common one \cite{Maxwell2018}. This means
that observations belonging to minority classes contribute less towards the
optimization process, translating into a bias towards majority classes. As
depicted in figure \ref{fig:oversampling_decision_function}a. As an example, a
trivial classifier can achieve 99\% overall accuracy on a binary dataset where
1\% of the observations belong to the minority class if it classifies all
observations as belonging to the majority class.

There are three different types of approaches to deal with the class imbalance
problem \cite{Fernandez2013,Kaur2019}:
\begin{enumerate}
	\item Cost-sensitive solutions. Introduces a cost matrix to the learning
	      phase with misclassification costs attributed to each class. Minority
	      classes will have a higher cost than majority classes, forcing the
	      algorithm to be more flexible and adapt better to predict minority
	      classes.
	\item Algorithmic level solutions. Specific classifiers are modified to
	      reinforce the learning on minority classes. Consists on the creation
	      or adaptation of classifiers.
	\item Resampling solutions. Rebalances the dataset's class distribution by
	      removing majority class instances and/or generating artificial
	      minority instances (see Figure
	      \ref{fig:oversampling_decision_function}). This is considered an
	      external  approach, where the intervention occurs before the learning
	      phase, benefitting from versatility and independency from the
	      classifier used.
\end{enumerate}

Within resampling approaches there are three subgroups of approaches
\cite{Fernandez2013,Kaur2019,Luengo2020}:
\begin{enumerate}
	\item Undersampling methods. They rebalance class distribution by removing
	      instances from the majority classes.
	\item Oversampling methods. Dataset is rebalanced by generating new
	      artificial instances belonging to the minority classes.
	\item Hybrid methods. Combination of both oversampling and undersampling,
	      resulting in the removal of instances in the majority classes and the
	      generation of artificial instances in the minority classes.
\end{enumerate}

Resampling methods can be further distinguished between non-informed and
heuristic (i.e., informed) resampling techniques
\cite{Fernandez2013,Luengo2020,Garcia2016}. The former consist of methods that
duplicate/remove a random selection of data points to set class distributions to
user-specified levels, and are therefore a simpler approach to the problem. The
latter consists of more sophisticated approaches that aim to perform
over/undersampling based on the points' contextual information within their data
space.

\begin{figure}[H]
	\centering
	\includegraphics[width=.85\linewidth]{../analysis/resampling_decision_function}
	\caption{Example of a linear Support Vector Machine's decision function (a) without
		resampling and (b) with resampling.
    }~\label{fig:oversampling_decision_function}
\end{figure}

In this paper, we propose the K-means SMOTE (K-SMOTE) \cite{Douzas2018}
oversampler to address the imbalanced learning problem in a multiclass context
for LULC classification using various remote sensing datasets. K-SMOTE's
efficacy is tested using different types of classifiers. To do so, we employ
both commonly used and state-of-the-art oversamplers as benchmarking methods:
Random oversampling (ROS), Synthetic Minority Oversampling Technique (SMOTE)
\cite{Chawla2002} and Borderline-SMOTE (B-SMOTE) \cite{Han2005}. Also as a
baseline score we include classification results without the employment of any
resampling method.

This paper is organized in 5 sections: section \ref{sec:sota} provides an
overview of the state-of-art, section \ref{sec:methodology} describes the
proposed methodology, section \ref{sec:results} covers the results and
discussion and section \ref{sec:conclusion} presents the conclusions taken from
this study.

\section{Imbalanced Learning Approaches} \label{sec:sota}

Existing methods that address imbalanced learning act on different stages. They
can act in the preprocessing step (Over/Undersampling and hybrid approaches), in
the learning process (cost-sensitive solutions) or in the algorithm itself (by
adapting existing algorithms and/or ensemble methods) \cite{Kaur2019}. In this
section, we focus on previous work related with resampling methods, while
providing a brief explanation of cost-sensitive and algorithmic level solutions.

All of the most common classifiers used for LULC classification tasks
\cite{Khatami2016, Gavade2019} are sensitive to class imbalance
\cite{Blagus2010}. Algorithm-based approaches typically focus on adaptations
based on ensemble classification methods \cite{Mellor2015} or common
non-ensemble based classifiers such as Support Vector Machines \cite{Shao2014}.
In \cite{Lee2016}, the reported results show that algorithm-based methods have
comparable performance to resampling methods.

Cost-sensitive solutions refer to changes in the importance attributed to each
instance through a cost matrix \cite{Huang2016,Cui2019,Dong2017}. A common cost
sensitive solution is found in \cite{Huang2016}. The authors use the inverse
class frequency (i.e., $1/|C_i|$) to give higher weight to minority classes. Cui
et al. \cite{Cui2019} extended this method by adding a hyperparameter $\beta$ to
class weights as $(1-\beta)/(1-\beta^{|C_i|})$. When $\beta=0$, no re-weighting
is done. When $\beta\rightarrow 1$, weights are the inverse of the frequency
class matrix. Another method \cite{Dong2017} explores adaptations of
Cross-entropy classification loss by adding different formulations of class
rectification loss.

Imbalanced Learning is most commonly addressed through data resampling in
machine learning in general and remote sensing in particular \cite{Feng2019}.
The generation of artificial instances (i.e., augmenting the dataset) based on
rare examples is done independently of any other classification and
preprocessing step. Once this step is applied, any standard machine learning
procedure can be applied. This simplicity makes resampling strategies
particularly appealing for any user interested in applying several classifiers
or maintaining a simple approach. Additionally, any of these methods can be
naturally applied to multiclass problems and particularly to LULC classification
tasks.

\subsection{Non-informed resampling methods}

There are two main non-informed resampling methods. Random Oversampling (ROS)
generates artificial observations through random duplication of rare instances.
This method is used in remote sensing \cite{Sharififar2019, Hounkpatin2018} for
its simplicity, even though its mechanism makes the classifier prone to
overfitting \cite{Krawczyk2016}. Hounkpatin et al. \cite{Hounkpatin2018} found
that using ROS returned worse results than keeping the original imbalance in
their dataset.

A few of the recent remote sensing studies employed Random Undersampling (RUS)
\cite{Ferreira2019}. This method, on the other hand, randomly removes
observations belonging to common classes. Although it's not as prone to
overfitting as ROS, it incurs into information loss by eliminating observations
from the majority class \cite{Feng2019}.

Another downfall of non-informed resampling methods is their performance-wise
inconsistency across classifiers. ROS' impact on the Indian Pines dataset was
found inconsistent between Random Forest Classifiers (RFC) and Support Vector
Machines (SVM) and lowered the predictive power of an artificial neural network
(ANN) \cite{Maxwell2018}. Similarly, RUS is found to generally lead to a lower
overall accuracy due to the associated information loss \cite{Maxwell2018}.

\subsection{Heuristic methods}

The methods presented in this section appear as a means to overcome the
insufficiencies found in non-informed resampling. They use either local or
global information to generate new, relevant, non-duplicated instances to
populate the minority classes and/or remove irrelevant instances from majority
classes. In a comparative analysis between over- and undersamplers' performance
for LULC classification \cite{Feng2018} using the rotation forest ensemble
classifier, authors found that oversampling methods consistently outperform
undersampling methods. Due to the scope of this study, heuristic undersampling
algorithms will not be analysed.

SMOTE \cite{Chawla2002} was the first heuristic oversampling algorithm to be
proposed and has been the most popular one since then, likely due to its fair
degree of simplicity and quality of generated data. It takes a random minority
class sample and introduces synthetic examples along the line segment that join
a random $k$ minority class nearest neighbor to the selected sample.
Specifically, a single synthetic sample $\overrightarrow{z}$ is generated within
the line segment of a randomly selected minority class observation
$\overrightarrow{x}$ and one of its $k$ nearest neighbors $\overrightarrow{y}$
such that $\overrightarrow{z} =
\alpha\overrightarrow{x}+(1-\alpha)\overrightarrow{y}$, where $\alpha$ is a
random floating point between 0 and 1, as shown in Figure
\ref{fig:smote_example}.

\begin{figure}[H]
	\centering
	\includegraphics[width=1\linewidth]{../analysis/smote_example}
	\caption{Example of SMOTE's data generation process.}
	\label{fig:smote_example}
\end{figure}

A number of studies implement SMOTE within the LULC classification context and
reported improvements on the quality of the trained predictors
\cite{Jozdani2019, Bogner2018}. Another study proposes an adaptation of SMOTE on
an algorithmic level for deep learning applications \cite{Zhu2020}. This method
combines both typical computer vision data augmentation techniques, such as
image rotation, scaling and flipping on the generated instances to populate
minority classes. Another algorithmic implementation is the variational
semi-supervised learning model \cite{Cenggoro2018}. It consists of a generative
model that allows learning from both labelled and unlabelled instances while
using SMOTE to balance the data.

Despite SMOTE's popularity, its drawbacks motivated the development of more
sophisticated oversampling algorithms \cite{Douzas2019}:
\begin{enumerate}
	\item Generation of noisy instances due to random selection of a minority
	      observation to oversample. The random selection of a minority
	      observation makes SMOTE oversampling prone to the amplification of
	      existing noisy data. In Figure \ref{fig:smote_example} it is possible
	      to observe a minority sample located within a cluster of majority
	      instances. Performing a linear interpolation between the noisy sample
	      $\overrightarrow{a}$ and one of its nearest neighbors
	      $\overrightarrow{b}$ will generate a noisy sample
	      $\overrightarrow{c}$. B-SMOTE \cite{Han2005} attempts to circumvent
	      the noisy data selection problem by performing a targeted selection of
	      instances close to the presumed class border, determined by the labels
	      of each sample's $k$ nearest neighbors. Alternatively, a sample will
	      discarded because it was deemed as either noisy, or being far from the
	      class boundary. Another algorithm that addresses this problem is
	      ADASYN \cite{HaiboHe2008}. It calculates a density distribution ratio
	      for each sample based on its $k$-nearest neighbors to determine the
	      number of synthetic observations to generate for each minority class
	      observation using the described SMOTE procedure.

	\item Generation of noisy instances due to the selection of the $k$ nearest
	      neighbors. In the event an observation (or a small number thereof) is
	      not noisy but is isolated from the remaining clusters, known as the
	      "small disjuncts problem" \cite{holte1989}, much like sample
	      $\overrightarrow{b}$ from Figure \ref{fig:smote_example}, the
	      selection of any nearest neighbor of the same class will have a high
	      likelihood of producing a noisy sample.

	\item Generation of nearly duplicated instances. Whenever the linear
	      interpolation is done between two observations that are close to each
	      other, the generated instance becomes very similar to its parents and
	      increases the risk of overfitting. G-SMOTE \cite{Douzas2019} attempts
	      to address both the $k$ nearest neighbor selection mechanism problem
	      as well as the generation of nearly duplicated instances problem. It
	      proposes a variation on SMOTE's data generation mechanism by
	      generating data within an oval geometry (instead of a line segment)
	      around the selected observation and the selected nearest neighbor. In
	      its turn, the $k$ nearest neighbors selection can include observations
	      from the remaining classes. To an extent, this algorithm can be
	      considered a generalized version of SMOTE, since under specific
	      hyperparameter definitions it replicates SMOTE's behavior.

	\item Generation of noisy instances due to the use of observations from two
	      different minority class clusters. Although an increased $k$ could
	      potentially avoid the previous problem, it can also lead to the
	      generation of artificial data between different minority clusters.
	      Cluster-based oversampling methods, as well as ADASYN, attempt to
	      address this problem. B-SMOTE \cite{Han2005} and G-SMOTE also address
	      this problem by allowing the interpolation to be performed with
	      majority class instances.
\end{enumerate}

Although no cluster-based oversampling approach applied within the remote
sensing domain was found in the literature, there are numerous methods to
consider. Cluster-based oversampling approaches introduce an additional layer to
SMOTE's selection mechanism, which is done according to the clustering process.
This is done to ensure both between-class data balance, but also ensure that the
data distribution within each class is preserved. The self-organizing map
oversampling (SOMO) \cite{Douzas2017} algorithm transforms the dataset into a
2-dimensional input, where the areas with the highest density of minority
samples are identified. SMOTE is then used to oversample each of the identified
areas separately. CURE-SMOTE \cite{Ma2017} applies a hierarchical clustering
algorithm (CURE) to discard isolated minority instances before applying SMOTE.
Although it avoids noise generation problems, it ignores within-class data
distribution. Another method \cite{Santos2015} uses K-means to cluster the
entire input space and applies SMOTE to clusters with the fewest observations,
regardless of their class label. The label of the generated observation is
copied from one of its parents. This method cannot ensure a balanced dataset
since class imbalance is not specifically addressed, but rather dataset
imbalance.

K-SMOTE~\cite{Douzas2018} avoids noisy data generation by modifying the data
selection mechanism. It employs $k$-means clustering to identify safe areas
using cluster-specific Imbalance Ratio (defined by
$\frac{count(C_{majority})}{count(C_{minority})}$) and determine the quantity of
generated samples per cluster based on a density measure. These samples are
finally generated using the SMOTE algorithm. The K-SMOTE's data generation
process is depicted in Figure~\ref{fig:kmeans_smote_example}. Note that the
number of samples generated for each cluster varies according to the sparsity of
each cluster (the sparser the cluster is, the more samples will be generated)
and a cluster is rejected if the cluster's IR surpasses the threshold.
Therefore, this method can be combined with any data generation mechanism, such
as G-SMOTE. Also K-SMOTE includes the SMOTE algorithm as a special case when the
number of clusters is set to one. Consequently, K-SMOTE is always guaranteed to
return results as good as or better than SMOTE.

\begin{figure}[H]
	\centering
	\includegraphics[width=1\linewidth]{../analysis/kmeans_smote_example}
	\caption{Example of K-SMOTE's data generation process. Clusters $A$,
		$B$ and $C$ are selected for
		oversampling, whereas cluster $D$ was rejected due to its
		high imbalance ratio. The oversampling is done using the SMOTE algorithm and
		the $k$ nearest neighbors selection only considers
		observations within the same cluster.}
	\label{fig:kmeans_smote_example}
\end{figure}

Although no other study was found to implement cluster-based oversampling,
another study \cite{Douzas2019rs} compared the performance of SMOTE, ROS,
ADASYN, B-SMOTE and G-SMOTE in a highly imbalanced LULC classification dataset.
The authors found that G-SMOTE consistently outperformed the remaining
oversampling algorithms regardless of the classifier used.

This paper main contributions are:
\begin{itemize}
	\item Testing these oversampling methods in multiple widely used LULC classification
	      datasets. Allows us to check for oversamplers' performance statistical
	      significance across datasets and report K-SMOTE's performance in benchmark LULC
	      datasets.
	\item Introducing a cluster-based oversampling algorithm within the remote sensing
	      domain, as well as comparing its performance with the remaining oversamplers in
	      a multiclass context.

\end{itemize}

\section{Methodology}\label{sec:methodology}

The purpose of this work is to understand the performance of K-SMOTE as opposed
to other popular and/or state-of-the-art oversamplers for LULC classification.
To do so, we employ 7 LULC datasets along with 3 evaluation metrics and 5
classifiers to evaluate the performance of oversamplers. In this section we
describe the datasets, evaluation metrics, oversamplers, classifiers and
software used as well as the procedure developed.

\subsection{Datasets}

The datasets used were extracted from publicly available hyperspectral scenes.
Information regarding each of these scenes is provided in this subsection. A
similar data preprocessing procedure was used for each scene: 1) Conversion of
each hyperspectral scene to a structured dataset and removal of instances with
no associated LULC class, 2) random sampling to maintain similar class
proportions on a sample of 10\% of each dataset and 3) removal of instances
belonging to a class with frequency lower than 20 or higher than 1000. This is
done to maintain the datasets to a practicable size due to computational
constraints, while conserving the relative LULC class frequencies and data
distribution. Table~\ref{tab:datasets_description} provides a description of the
final datasets used for this work.

\pgfplotstabletypeset[
	begin table=\begin{longtable},
		end table=\end{longtable},  col sep=comma, header=true,
	columns={Dataset,Features,Instances,Minority instances,Majority instances,IR, Classes}, string type, every head row/.style={before row=\toprule, after row=\midrule\endhead},
	every last row/.style={after row=\bottomrule \caption{\label{tab:datasets_description}
				Description of the datasets used for this experiment.}}
]{../analysis/datasets_description.csv}

\subsubsection*{Indian Pines} 

The Indian Pines scene~\cite{Baumgardner2015} was collected on June 12, 1992 and
consists of AVIRIS hyperspectral image data covering the Indian Pine Test Site
3, located in North-western Indiana, USA. As a subset of a larger scene, it is
composed of $145 \times 145$ pixels (see Figure~\ref{fig:indian_pines}) and 220
spectral reflectance bands in the wavelength range 400 to 2500 nanometers.
Approximately two thirds of this scene is composed by agriculture and the other
third is composed of forest and other natural perennial vegetation.
Additionally, the scene also contains low density buildup areas.

\subsubsection*{Pavia Centre and University}

Both Pavia Centre and University scenes were acquired by the ROSIS sensor. These
scenes are located in Pavia, northern Italy. Pavia Centre is a $1096 \times
1096$ pixels image with 102 spectral bands, whereas Pavia University is a $610
\times 610$ pixels image with 103 spectral bands.  Both images have a
geometrical resolution of 1.3 meters and their ground truths are composed of 9
classes each (see Figures~\ref{fig:pavia_centre}
and~\ref{fig:pavia_university}).

\subsubsection*{Salinas and Salinas-A}

These scenes were collected by the AVIRIS sensor over Salinas Valley, California
and contain at-sensor radiance data. Salinas is a $512 \times 217$ pixels image
with 224 bands and 16 classes regarding vegetables, bare soil and vineyard
fields (see Figure~\ref{fig:salinas}). Salinas-A, a subscene of Salinas,
comprises $86 \times 83$ pixels and contains 6 classes regarding vegetables (see
Figure~\ref{fig:salinas_a}). These scenes have a geometrical resolution of 3.7
meters.

\subsubsection*{Botswana}

The Botswana scene was acquired by the Hyperion sensor on the NASA EO-1
satellite over the Okavango Delta, Botswana in 2001-2004 at a 30m spatial
resolution. Data preprocessing was performed by the UT Center for Space
Research. The scene comprises a $1476 \times 256$ pixels with 145 bands and 14
classes regarding land cover types in seasonal and occasional swamps, as well as
drier woodlands (see figure~\ref{fig:botswana}).

\subsubsection*{Kennedy Space Center}

The Kennedy Space Center scene was acquired by the AVIRIS sensor over the
Kennedy Space Center, Florida, on March 23, 1996. Out of the original 224 bands,
water absorption and low SNR bands were removed and a total of 176 bands at a
spatial resolution of 18m are used. The scene is a $512 \times 614$ pixel image
and contains a total of 16 classes (see figure~\ref{fig:kennedy_space_center}).

\begin{figure}[H]
	\centering
	\begin{subfigure}{.24\textwidth}
		\centering
		\captionsetup{skip=12pt}
		\includegraphics[height=1.5\linewidth]{../analysis/indian_pines}
		\subcaption{{\medbreak}}\label{fig:indian_pines}
	\end{subfigure}
	\begin{subfigure}{.24\textwidth}
		\centering
		\includegraphics[height=1.5\linewidth]{../analysis/pavia_centre}
		\subcaption{{\medbreak}}\label{fig:pavia_centre}
	\end{subfigure}
	\begin{subfigure}{.24\textwidth}
		\centering
		\includegraphics[height=1.5\linewidth]{../analysis/pavia_university}
		\subcaption{{\medbreak}}\label{fig:pavia_university}
	\end{subfigure}

	\begin{subfigure}{.24\textwidth}
		\centering
		\includegraphics[height=1.5\linewidth]{../analysis/salinas}
		\subcaption{{\medbreak}}\label{fig:salinas}
	\end{subfigure}
	\begin{subfigure}{.24\textwidth}
		\centering
		\includegraphics[height=1.5\linewidth]{../analysis/salinas_a}
		\subcaption{{\medbreak}}\label{fig:salinas_a}
	\end{subfigure}
	\begin{subfigure}{.24\textwidth}
		\centering
		\includegraphics[height=1.5\linewidth]{../analysis/botswana}
		\subcaption{{\medbreak}}\label{fig:botswana}
	\end{subfigure}
	\begin{subfigure}{.24\textwidth}
		\centering
		\includegraphics[height=1.5\linewidth]{../analysis/kennedy_space_center}
		\subcaption{{\medbreak}}\label{fig:kennedy_space_center}
	\end{subfigure}
	\caption{Gray scale visualization of a band (top row) and ground truth (bottom row) of
		each scene used in this study. (a) Indian Pines, (b) Pavia Centre, (c) Pavia
		University, (d) Salinas, (e) Salinas A, (f) Botswana, (g) Kennedy Space Center
    }\label{fig:scenes}
\end{figure}

\subsection{Evaluation Metrics}
Most of the satellite-based LULC classification studies (nearly 80\%) employ
\textit{Overall Accuracy} (OA) and the \textit{Kappa Coefficient}
\cite{Gavade2019}. Although, some authors argue that both evaluation metrics,
even when used simultaneously, are insufficient to fully address the area
estimation and uncertainty information needs \cite{Olofsson2013,Pontius2011}.
Other metrics like User's Accuracy (or \textit{Precision}) and Producer's
Accuracy (or \textit{Recall}) are also common metrics to evaluate per-class
prediction power. These metrics consist of ratios employing the True and False
Positives (\textit{TP} and \textit{FP}, number of correctly/incorrectly
classified observations of a given class) and True and False Negatives
(\textit{TN} and \textit{FN}, number of correctly/incorrectly classified
observations as not belonging to a given class). These metrics are formulated as
$Precision = \frac{TP}{TP+FP}$ and $Recall = \frac{TP}{TP+FN}$. While metrics
like OA and \textit{Kappa Coefficient} are significantly affected by imbalanced
class distributions, \textit{F-Score} is less sensitive to data imbalance and a
more appropriate choice for performance evaluation \cite{Jeni2013}.

The datasets used present significantly high IRs (see Table
\ref{tab:datasets_description}). Therefore, it is especially important to
attribute equal importance to the predictive power of all classes, which does
not happen with OA and \textit{Kappa Coefficient}. In this study, we employ 3
evaluation metrics: 1) \textit{G-mean}, since it is not affected by skewed class
distributions, 2) \textit{F-Score}, as it proved to be a more appropriate metric
for this problem when compared to other commonly used metrics \cite{Jeni2013},
and 3) \textit{Overall Accuracy}, for discussion purposes.

\begin{itemize}
	\item The \textit{G-mean} consists of the geometric mean of
	      $Specificity = \frac{TN}{TN + FP}$ and \textit{Sensitivity} (also known as \textit{Recall}). For multiclass problems, The
	      \textit{G-mean} is expressed as:

	      $$\textit{G-mean} = \sqrt{ \overline{Sensitivity} \times
			      \overline{Specificity}}$$

	\item \textit{F-score} is the harmonic mean of \textit{Precision} and
	      \textit{Recall}. The \textit{F-score} for the multi-class case can
	      be calculated using their average per class values \cite{He2009}:

	      $$\textit{F-score}=2\frac{\overline{Precision} \times \overline{Recall}}{\overline{Precision} +
			      \overline{Recall}}$$

	\item \textit{Overall Accuracy} is the number of correctly classified observations
	      divided by the total amount of observations. Having \( c \) as the label of the
	      various classes, \textit{Accuracy} is given by the following formula:

	      $$\textit{Accuracy} = \frac{ \sum\limits_{c}{ \text{TP}_{c} } }{
			      \sum\limits_{c}{ (\text{TP}_{c}  + \text{FP}_{c}) } } $$

\end{itemize}

\subsection{Machine Learning Algorithms}
The assess the quality of the K-SMOTE algorithm, five other oversampling
algorithms were used for benchmarking. ROS and SMOTE were chosen for their
simplicity and popularity. ADASYN and B-SMOTE were chosen for their popular
variations of the SMOTE algorithm. Finally, G-SMOTE was chosen for being a
state-of-the-art oversampler as it was found to outperform all of the other
oversamplers in \cite{Douzas2019rs}. Additionally we include the classification
results of no oversampling (NONE) as a baseline.

To assess the performance of each oversampler, we use the classifiers Logistic
Regression (LR) \cite{McCullagh1989}, K-Nearest Neighbors (KNN)
\cite{Cover1967}, Decision Tree (DT) \cite{Salzberg1994}, Gradient Boosting
Classifier (GBC) \cite{Friedman2001} and Random Forest (RF) \cite{Liaw2002}.
This choice was based on the classifiers' popularity for LULC classification,
learning type and training time \cite{Maxwell2018,Gavade2019}.

\subsection{Experimental Procedure}
The procedure for the experiment reported in this study is similar to the one
proposed in \cite{Douzas2019rs}. We start by defining a hyperparameter search
grid, where a list of possible values for each relevant hyperparameter in both
classifiers and oversamplers is stored. Based on this search grid, all possible
combinations of oversamplers, classifiers and hyperparameters are formed.
Finally, for each dataset we employ a $k$-fold cross-validation strategy where
$k=5$ to train each model defined and save the averaged scores of each split.

Each combination of oversampler, classifier and parameters definition is fit 5
times (once for each fold) per dataset. Each time, an oversampler will use the
training set ($80\%$ of the dataset) to generate a set with artificial data,
which is appended to the original training set in order to generate a training
dataset with the exact same number of observations for each class. The newly
formed training dataset is used to train the classifier and the test set ($20\%$
of the dataset, the remaining fold) is used to evaluate the performance of the
classifier. The evaluation scores are then averaged over the 5 times the process
is repeated. The range of hyperparameters used are shown in table
\ref{tab:grid}.

\begin{table}[H]
	\centering
	\begin{tabular}{lll}
		\toprule
		Classifier       & Hyperparameters      & Values                            \\
		\midrule
		LR               & maximum iterations   & 10000                             \\
		KNN              & \# neighbors  & {3, 5, 8}                            \\
		RF               & maximum depth        & {None, 3, 6}                      \\
		                 & \# estimators & {50, 100, 200}                         \\
		\toprule
		Oversampler      &                      &                                   \\
		\midrule
		K-SMOTE          & \# neighbors  & {3, 5}                            \\
		                 & \# clusters (as \% of number of observations)   & {1$^*$, 0.1, 0.3, 0.5, 0.7, 0.9}      \\
                         & Exponent of mean distance & {auto, 2, 5, 7}       \\
                         & IR threshold  & {auto, 0.5, 0.75, 1.0}            \\
		SMOTE            & \# neighbors  & {3, 5}                            \\
		BORDERLINE SMOTE & \# neighbors  & {3, 5}                            \\
		\bottomrule
	\end{tabular}
    \caption{\label{tab:grid}Hyper-parameters grid. $^*$~One cluster is generated in total, a corner
        case that mimics the behavior of SMOTE}
\end{table}

\subsection{Software Implementation}~\label{sec:implementation}

The experiment was implemented using the Python programming language, using the
\href{https://scikit-learn.org/stable/}{Scikit-Learn}~\cite{Pedregosa2011},
\href{https://imbalanced-learn.org/en/stable/}{Imbalanced-Learn}~\cite{JMLR:v18:16-365},
\href{https://geometric-smote.readthedocs.io/en/latest/?badge=latest}{Geometric-SMOTE},
\href{https://cluster-over-sampling.readthedocs.io/en/latest/?badge=latest}{Cluster-Over-Sampling}
and \href{https://research-learn.readthedocs.io/en/latest/?badge=latest}{Research-Learn} libraries.
All functions, algorithms, experiments and results are provided at the
\href{https://github.com/joaopfonseca/publications/tree/master/remote-sensing-kmeans-smote}{GitHub
repository of the project}.

\section{Results}~\label{sec:results}
When evaluating the performance of an algorithm across multiple datasets, it is
generally recommended to avoid direct score comparisons and use classification
rankings instead~\cite{demvsar2006}. This is done by assigning a ranking to
oversamplers based on the different combinations of classifier, metric and
dataset used. These rankings are also used for the statistical analyses
presented in Section~\ref{sec:statistical_analysis}.

The rank values are assigned based on the mean validation scores resulting from
the experiment described in Section~\ref{sec:methodology}. The averaged ranking
results are computed over 3 different initialization seeds and a 5 fold cross
validation scheme, returning a float value within the interval $[1,5]$. The mean
rankings are presented in Table~\ref{tab:mean_sem_ranking} and
Figure~\ref{fig:mean_rankings_bar_chart}.

\begin{figure}[H]
	\centering
	\includegraphics[width=.6\linewidth]{../analysis/mean_rankings_bar_chart}
	\caption{Mean ranking of oversamplers across datasets.
    }\label{fig:mean_rankings_bar_chart}
\end{figure}

The mean ranking results show that K-SMOTE consistently presents the best
results for every classifier and performance metric used. This is visually
depicted in Figure~\ref{fig:mean_rankings_bar_chart}. The quantitative results
of this analysis is presented in Table~\ref{tab:mean_sem_ranking}. In addition
to its better performance, in most cases K-SMOTE's mean ranking has a lower
standard deviation than any of the remaining methods, and particularly when
opposed to SMOTE (the best performing benchmark method).

\pgfplotstabletypeset[
	begin table=\begin{longtable},
		end table=\end{longtable}, col sep=comma, header=true,
	columns={Classifier,Metric,NONE,ROS,SMOTE,B-SMOTE,K-SMOTE}, string type, every head row/.style={before row=\toprule, after row=\midrule\endhead},
	every last row/.style={after row=\bottomrule \caption{\label{tab:mean_sem_ranking}
				Results for mean ranking of oversamplers across datasets.}}
]{../analysis/mean_sem_ranking.csv}

The mean percentage difference among K-SMOTE and SMOTE is presented in
Figure~\ref{fig:mean_score_improvement_heatmap}. It is calculated as the score
difference among the test (K-SMOTE) and control (SMOTE) oversampler, divided by
the control oversampler's score. K-SMOTE's average performance improves
classification performance of up to $1.9\%$ and outperforms all other methods,
with the exception of two situations when using the G-mean evaluation metric.

\begin{figure}[H]
	\centering
	\includegraphics[height=.4\linewidth]{../analysis/mean_score_improvement_heatmap}
    \caption{Mean score improvement (percentage difference) of the proposed method versus
        SMOTE across datasets.
        }\label{fig:mean_score_improvement_heatmap}
\end{figure}

The mean cross-validation scores are shown in Table~\ref{tab:mean_sem_scores}.
Considering the disparity of performance scores across datasets, the results
presented in this table may not be as informative as the scores for each
dataset, presented in Table~\ref{tab:cross_validation_scores}. K-SMOTE's
performance is the highest in most classifier/metric combinations and datasets,
showing more inconsistency on the Indian Pines and Kennedy Space Center
datasets.

\pgfplotstabletypeset[
	begin table=\begin{longtable},
		end table=\end{longtable},  col sep=comma, header=true,
	columns={Classifier,Metric,NONE,ROS,SMOTE,B-SMOTE,K-SMOTE}, 
    string type, every head row/.style={before row=\toprule, after row=\midrule\endhead},
	every last row/.style={after row=\bottomrule \caption{\label{tab:mean_sem_scores}
				Mean cross-validation scores of oversamplers.}}
]{../analysis/mean_sem_scores.csv}

The performance of both oversamplers and classifiers is generally dependent on
the dataset being used. Although both absolute and relative scores between the
different oversamplers are dependent on the choice of metric and classifier,
K-SMOTE's relative performance is consistent across datasets and generally
outperforms the remaining oversampling methods.

\pgfplotstabletypeset[
	begin table=\begin{longtable},
		end table=\end{longtable},
	col sep=comma,
	header=true,
	columns={Dataset, Classifier,Metric,NONE,ROS,SMOTE,B-SMOTE,K-SMOTE}, 
    string type,
	every head row/.style={before row=\toprule, after row=\midrule\endhead},
	every last row/.style={after row=\bottomrule
			\caption{\label{tab:cross_validation_scores}Mean cross-validation scores of oversamplers
        for each dataset. Legend: IP \- Indian Pines, KSC \- Kennedy Space Center, PC \- Pavia Center,
        PU \- Pavia University, SA \- Salinas A.}}
]
{../analysis/wide_optimal.csv}

\subsection{Statistical Analysis}~\label{sec:statistical_analysis}

The experiment's multi-dataset context was used to perform both a Friedman
test~\cite{friedman1937use}. Table~\ref{tab:friedman_test} shows the results
obtained in the Friedman test performed, where the null hypothesis is rejected
in all cases. Consequently, the Holm-Bonferroni comparison method (Holm's
method)~\cite{holm1979simple} is used for post-hoc analysis.

\pgfplotstabletypeset[
	begin table=\begin{longtable},
		end table=\end{longtable},  col sep=comma, header=true,
	columns={Classifier,Metric,p-value,Significance}, string type, every head row/.style={before row=\toprule, after row=\midrule\endhead},
	every last row/.style={after row=\bottomrule \caption{\label{tab:friedman_test}
				Results for Friedman test. Statistical significance is tested at a level of
				$\alpha = 0.05$. The null hypothesis is that there is no difference in the
                classification outcome across oversamplers.}}
]{../analysis/friedman_test.csv}

The results of the Holm's method are shown in Table~\ref{tab:holms_test}. Even though K-SMOTE
outperforms the remaining oversamplers, the datasets' inherent high prediction scores make the
rejection of this null hypothesis particularly difficult.

\pgfplotstabletypeset[
	begin table=\begin{longtable},
		end table=\end{longtable},  col sep=comma, header=true,
    columns={Classifier,Metric,NONE,ROS,SMOTE,B-SMOTE}, 
    string type, 
    every head row/.style={before row=\toprule, after row=\midrule\endhead},
	every last row/.style={after row=\bottomrule 
        \caption{\label{tab:holms_test} Adjusted p-values using the Holm's method. Bold values are
            statistically significant at a level of $\alpha=0.05$. The null hypothesis is that the
    test method does not perform better than the control method.}}
]{../analysis/holms_test.csv}

\section{Conclusion}~\label{sec:conclusion} 
This research paper was motivated by the difficulty posed in classifying rare
classes in Land Use/Land Cover tasks. A number of existing methods to address
this problem (known as imbalanced learning) was identified and their caveats
were exposed.  Typically, these methods are not only difficult to implement,
they are also context dependent. We focused on oversampling methods due to their
widespread usage, easy implementation and flexibility.  Specifically, this paper
demonstrated the efficacy of a recent oversampler, K-Means SMOTE, applied in a
multi-class context for Land Cover Classification tasks. This was done with
sampled data from seven well known and naturally imbalanced datasets: Indian
Pines, Pavia Centre, Pavia University, Salinas, Salinas A, Botswana and Kennedy
Space Center. The experiment comprised a hyper-parameter search in order to tune
each algorithm to its specific use case. For each combination of dataset,
oversampler and classifier, the results of every classification task was
averaged across a 5 fold stratification strategy with 3 different initialization
seeds, resulting in a mean validation score of 15 classification tasks.  The
optimal mean validation score of each combination was then used to perform the
analyses presented in this report.

In most cases, classification tasks using K-SMOTE led to better results than
using the original, unmodified, imbalanced data. More importantly, we found that
K-Means SMOTE is always better or equal than the second best oversampling
method. K-SMOTE's performance was independent from both the classifier and
performance metric under analysis. In general, K-SMOTE shows a higher
performance among the non tree-based classifiers employed, when compared to the
remaining oversamplers. Although these findings are case dependent, they are
consistent with the results presented in~\cite{Douzas2018}. The proposed method
also had the most consistent results across datasets, since it had the lowest
standard deviations across datasets in most cases for both analyses, either
based on ranking or mean cross-validation scores.

The proposed algorithm is an extension of the original SMOTE algorithm. In fact,
the SMOTE algorithm represents a corner case of K-SMOTE i.e. when the number of
clusters equals to 1. Its data selection phase differs from the one used in
SMOTE and Borderline SMOTE, providing artificially augmented datasets with less
noisy data than the commonly used methods. This allows the training of
classifiers with better defined decision boundaries, especially in the most
important regions of the data space (the ones populated by a higher percentage
of minority class instances).

As stated previously, the usage of this oversampler is technically simple. It
can be applied to any classification problem relying on an imbalanced dataset,
alongside any classifier. K-SMOTE is available as an open source implementation
for the Python programming language (see Subsection~\ref{sec:implementation}).
Consequently, it can be a useful tool for both remote sensing researchers and
practitioners.

\bibliography{references}
\bibliographystyle{apalike}

\end{document}
